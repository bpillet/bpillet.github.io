\documentclass{article}
\usepackage[color]{teaching}
	
\begin{document}
\dochead{Licence 2 --- Mathématiques}{Fonctions de plusieurs variables}{Correction CC 1}

\exercice{}

\begin{enumerate}
\item Faux !
Par exemple $\Oo = \emptyset$ est un ouvert et alors $\Oo \cup A = A$. Donc, dans ce cas, $\Oo \cup A$ n'est un ouvert de $X$ que si $A$ est ouvert de $X$.

Autre exemple, dans $\R$ prendre $\Oo = ]0,1[$ et $A = \{1\}$. Alors $\Oo \cup A = ]0,1]$ qui n'est pas ouvert.
\item Faux !
Par exemple $\Ff = X$ est un fermé de $X$ et alors $\Ff \cap B = B$. Donc, dans ce cas, $\Ff \cap B$ n'est un fermé de $X$ que si $B$ est fermé.

Autre exemple, dans $\R$ prendre $\Ff = [-2,2]$ et $B = ]-1,5]$. Alors $\Ff \cap B = ]-1,2]$ qui n'est pas fermé.
\end{enumerate}

\exercice{}

\begin{enumerate}
\item $f$ est définie sur $\R^2$. De plus sur $\R^2 \setminus \{(0,0)\}$ c'est une fraction rationnelle dont le dénominateur ne s'annule pas. C'est donc une fonction continue sur $\R^2 \setminus \{(0,0)\}$.

Reste à regarder si elle est continue sur en $(0,0)$.

\textbf{1er méthode :} 
En coordonnées polaires
\begin{align*}
x &= r\cos(\theta)\\
y &= r\sin(\theta)
\end{align*}	
 avec $r>0$ et $\theta \in [0,2\pi[$.
 
On a alors
\[
f(x,y) = \dfrac{(r\cos(\theta))^m(r\sin(\theta))^n}{r^4(\cos^4(\theta)+\sin^4(\theta))}
= r^{m+n-4} \dfrac{\cos^m(\theta)\sin^n(\theta)}{\cos^4(\theta)+\sin^4(\theta)}
\]
Première remarque 
\[
\cos^4(\theta)+\sin^4(\theta) = (\cos^2(\theta) + \sin^2(\theta))^2 - 2\cos^2(\theta)\sin^2(\theta) = 1 - 2(\cos(\theta)\sin(\theta))^2
\]
Or $\cos(\theta)\sin(\theta) = \demi\sin(2\theta) \in [-\demi,\demi]$. Donc $2(\cos(\theta)\sin(\theta))^2 \in  [-\demi,\demi]$

Bilan $\cos^4(\theta)+\sin^4(\theta) > \demi$ et donc la fraction
\[
\dfrac{\cos^m(\theta)\sin^n(\theta)}{\cos^4(\theta)+\sin^4(\theta)}
\] 
est bornée.

On remarque ensuite que : \textbf{Si} $m+n-4 > 0$ \textbf{alors} la limite de $f$ quand $(x,y) \to (0,0)$ vaut $0$ car $r^{m+n-4} \to 0$ et la fraction est bornée.

Reste à vérifier la réciproque~: si $m+n \leq 4$ alors $f$ est discontinue.

Par exemple si $\theta = \pi/4$ alors  
\[
\dfrac{\cos^m(\theta)\sin^n(\theta)}{\cos^4(\theta)+\sin^4(\theta)} = 2 \times \left(\dfrac{\sqrt{2}}{2}\right)^{m+n} \neq 0
\]
Et donc 
\[
\underset{r\to 0, \theta = \pi/4}{\lim} f(x,y) \neq 0
\]
Donc $f$ ne peut pas être continue en $0$.


\textbf{2ème méthode :} On rappelle que $\Vert (x,y) \Vert^2 = x^2 + y^2$ et que par définition $\Vert (x,y) \Vert \to 0$ \ssi $(x,y) \to 0$.
 
 \begin{align*}
 |x| &\leq  \Vert (x,y) \Vert\\
 |y| &\leq  \Vert (x,y) \Vert\\
 x^4+ y^4 &\geq (x^2+y^2)^2 \geq  \Vert (x,y) \Vert^4
 \end{align*}
 Bilan
 \[
\vert f(x,y) \vert  \leq \dfrac{\Vert (x,y) \Vert^m\Vert (x,y) \Vert^n}{\Vert (x,y) \Vert^4} = \Vert (x,y) \Vert^{m+n-4}
 \]
Ce qui permet de conclure que $f$ est continue en $0$ dès lors que $m+n-4 >0$.

\item Les dérivées partielles en $0$ sont définies comme limite du taux d'accroissement~:
\[
\dpp{f}{x}(0,0)= \lim_{h \to 0} \dfrac{1}{h} \left( f(h,0)  - f(0,0) \right) = \lim_{h \to 0} \dfrac{1}{h} \left(\dfrac{h^m0^n}{h^4} \right)
\]
Cette limite existe et vaut $0$ dès lors que $n>0$ ou ($n=0$ et $m>5$). Dans le cas $n=0$ et $m=5$, cette limite vaut $1$. Elle n'existe pas ou n'est pas finie dans tous les autres cas.
De même pour la dérivée par rapport à $y$ (ou par un argument de symétrie (à rédiger en exercice))
\[
\dpp{f}{y}(0,0) = \left\lbrace
\begin{array}{cc}
0 & m>0 \\ 
1 & m=0, n=5 \\ 
0 & m=0, n> 5
\end{array} 
\right.
\]
et n'existe pas ou n'est pas finie dans les autres cas.

En conclusion les dérivées partielles existent en $(0,0)$ si $m,n>0$ ou si $\max (m,n)\geq 5$. En particulier pour $m=n=1$, par ce qui précède $f$ n'est pas continue en $(0,0)$, pourtant des dérivée partielles d'ordre $1$ existent et on a~:
\[
\dpp{f}{x}(0,0) = \dpp{f}{y}(0,0) = 0
\]

\item La fonction $f$ admet des dérivées partielles à tout ordre en dehors de $(0,0)$. On les obtient par les formules usuelles de dérivation~:
\[
\dpp{f}{x}(x,y) = \dfrac{mx^{m-1}y^n(x^4+y^4) - 4x^{m+3}y^n}{(x^4+y^4)^2} = 
\dfrac{(m-4)x^{m+3}y^n +mx^{m-1}y^{n+4}}{(x^4+y^4)^2}
\]
On vérifie prudemment que cette formule donne bien le résultat même dans le cas $m=0$ et $m=1$. Et de même par rapport à $y$~:
\[
\dpp{f}{y}(x,y) = \dfrac{(n-4)y^{n+3}x^n +ny^{n-1}x^{m+4}}{(x^4+y^4)^2}
\]
En passant en polaire comme précédemment, on obtient
\[
\dpp{f}{x}(x,y) = r^{m+n+3-2\times 4} \dfrac{(m-4)\cos^{m+3}\sin^n +m\cos^{m-1}\sin^{n+4}}{(\cos^4+\sin^4)^2}
\]
On se restreindra à partir de maintenant au cas $m,n>0$ comme demandé dans l'énoncé. Dès lors cette fonction tend vers $0$ \ssi $m+n-5 >0$. Un raisonnement symétrique pour $y$ permet de conclure que les dérivées partielles de $f$ sont continues en $(0,0)$ \ssi $m+n >5$. Comme $f$ est clairement de classe $\Cc^1$ hors de $(0,0)$. On peut dire que $f$ est de classe $\Cc^1$ \ssi $m+n >5$.
\item Si la différentielle de $f$ existe en $(0,0)$, elle est donnée par les dérivées partielles en $(0,0)$ dès lors
\[
Df(0,0)(h,k) = h \dpp{f}{x}(0,0) + k \dpp{f}{y}(0,0) = 0
\]
\item Si est différentiable en $(0,0)$, alors elle est continue et donc par le point (1), on a $m+n \geq 5$. Cependant elle n'est de classe $\Cc^1$ que si $m+n > 5$ donc son caractère différentiable n'implique pas son caractère $\Cc^1$.
\end{enumerate}

\exercice{}
On suppose $\E$ de dimension finie.

$\phi$ est continue car linéaire et on a
\[
\phi(a+h) = \phi(a) + \phi(h)
\]
par linéarité de $\phi$, il s'en suit donc que en posant $L = \phi : \E \rightarrow \E$ et $\epsilon$ la fonction nulle de $\E \rightarrow \E$.

L'égalité du dessus se réécrit donc
\[
\phi(a+h) = \phi(a) + L(h) + \Vert h \Vert_\E \epsilon(h)
\]
avec $\lim_{h \to 0} \epsilon(h) = 0$ et $L$ linéaire continue donc $L$ est la différentielle de $\phi$ en $a$.

Ainsi
\[
D\phi(a)(h) = L(h) = \phi(h)
\]

\exercice{}
On vérifie les axiomes de norme~:
\begin{enumerate}[(a)]
\item $N : \R[X] \rightarrow \R^+$
\item \textit{homogénéité positive}~: $\forall \lambda \in \R, \forall p \in \R[X], N(\lambda p) = |\lambda| N(p)$
\item \textit{définie}~: $\forall p \in \R[X], N(p) = 0 \ssi p=0$
\item \textit{Inégalité triangulaire (pour les normes)}~: $\forall p,q \in \R[X], N(x+y) \leq N(p) + N(q)$
\end{enumerate}

\begin{enumerate}[{\bf (a)}]
\item Il est clair que $N$ est définie sur $\R[X]$ tout entier et à valeur positives.
\item Soit $\lambda \in \R$ et soit $p \in \R[X]$, alors en posant
\[
p(X) = a_0 + a_1X + \cdots + a_n X^n
\]
on a
\[
(\lambda p)(X) = (\lambda a_0) + (\lambda a_1)X + \cdots + (\lambda a_n) X^n
\]
et donc \[
N(\lambda p) = \sup_{i\in \N}( \vert \lambda a_i \vert) = \sup_{i\in \N}( \vert \lambda \vert \vert a_i \vert) = \vert \lambda \vert \sup_{i\in \N}(  \vert a_i \vert) = \vert \lambda \vert N(p)
\]
Ceci étant vrai quelque soit $\lambda \in \R$ et $p \in \R[X]$ on a bien montré que $N$ vérifie l'axiome d'homogénéité (b).
\item Il est clair que si $p =0$ est le polynôme nul alors $N(p) = 0$. Réciproquement, soit $p \in \R[X]$ tel que $N(p) = 0$. Posons 
\[
p(X) = a_0 + a_1X + \cdots + a_n X^n
\]
Alors on en déduit que $\sup_{i\in \N}( \vert a_i \vert) = 0$ donc pour tout $i \in \N$, $\vert a_i \vert  = 0$ donc $p = 0$.
On a bien montré que pour $p \in \R[X]$, $N(p) = 0 \ssi p = 0$.
\item Soient $p,q \in \R[X]$, on pose
\begin{align*}
p(X) &= a_0 + a_1X + \cdots + a_n X^n\\
q(X) &= b_0 + b_1X + \cdots + b_n X^n
\end{align*}
quitte à poser $b_n, b_{n-1}, \cdots, b_k = 0$ si le degré de $q$ est inférieur à $k$.
Alors $p+q$ est donné par
\[
(p+q)(X) = (a_0+b_0) + (a_1+b_1)X + \cdots + (a_n+b_n) X^n
\]
dès lors, en utilisant l'inégalité triangulaire pour la valeur absolue et la propriété "la borne supérieure d'une sommes est plus grande que la somme des bornes supérieures" on trouve
\[
N(p+q) = \sup_{i \in \N} \vert a_i + b_i \vert \leq \sup_{i \in \N} (\vert a_i \vert  + \vert b_i \vert ) \leq \sup_{i \in \N}\vert a_i \vert  + \sup_{j \in \N}\vert b_j \vert = N(p) + N(q)
\]
Ceci étant vrai quelque soient $p$ et $q$ dans $\R[X]$. On a donc bien montré l'inégalité triangulaire.
\end{enumerate}

Bilan $N$ est bien une norme sur $\R[X]$.

\paragraph{}
La fonction $\psi$ est définie et continue sur $\R[X]$ comme fonction polynomiale (n'a rien avoir avec le fait que les éléments de $\R[X]$ sont des polynômes ; c'est simplement la fonction $u \mapsto u^2 + 5u$ qui est polynomiale, et donc les propriétés usuelles de continuité s'appliquent).

Soit $p,h \in \R[X]$, considérons
\[
\psi(p + h) = (p+h)^2 + 5(p+h) = p^2 + 5p + 2ph + 5h + h^2 = \psi(p) + L(h) + h^2
\]
où $L(h) = (2p+5)h$. Il est clair que $L$ est linéaire et continue par rapport à $h$. Reste à montrer que $h^2 = N(h)\epsilon(h)$ avec $\epsilon$ qui a pour limite $0 \in \R[X]$ quand $h \to 0$.

Pour cela, on peut admettre\footnote{Cf annexe pour une preuve} que $N(h^2)/N(h) \to 0$ quand $h \to 0$ et $h \neq 0$. Dès lors on a bien  $D\psi(p)h = L(h)$.

\exercice{}
On supposera $z$ de classe $\Cc^1$.

On a une égalité de deux fonctions de $x,y$~:
\[
x^2+y^2+z(x,y)^2 = \phi(x + y + z(x,y))
\]
Comme ces deux fonctions sont de classe $\Cc^1$ sur $\R$ comme composés et sommes de fonctions de classe $\Cc^1$ sur $\R$ on peut dériver l'égalité par rapport à $x$ et $y$ pour obtenir
\begin{align*}
2x + 2z\dpp{z}{x}(x,y) &= \phi'(x+y+z)\left(1+ \dpp{z}{x}(x,y)\right)\\
2y + 2z\dpp{z}{y}(x,y) &= \phi'(x+y+z)\left(1+ \dpp{z}{y}(x,y)\right)\\
\end{align*}
où $z$ signifie réellement $z(x,y)$.

En faisant la différence des deux equations on obtient
\[	
2(x-y) + \left(2z - \phi'(x+y+z)\right)\left(\dpp{z}{x}(x,y)-\dpp{z}{y}(x,y)\right) = 0
\]
D'autre part, en multipliant la première par $\dpp{z}{y}(x,y)$ et la seconde par $\dpp{z}{x}(x,y)$ et en faisant la différence, on obtient~:
\[
2x \dpp{z}{y}(x,y) - 2y\dpp{z}{x}(x,y) = \phi'(x+y+z)\left(\dpp{z}{x}(x,y)-\dpp{z}{y}(x,y)\right)
\]

En regroupant, il s'en suit que
\[
2(x-y) + 2z\left(\dpp{z}{x}(x,y)-\dpp{z}{y}(x,y)\right) = \phi'(x+y+z)\left(\dpp{z}{x}(x,y)-\dpp{z}{y}(x,y)\right)
=2x \dpp{z}{y}(x,y) - 2y\dpp{z}{x}(x,y)
\]
Et donc
\[
x- y = (x-z)\dpp{z}{y}(x,y) + (z-y)\dpp{z}{x}(x,y)
\]

\appendix
\pagebreak
\subsection*{Annexe exercice 3}
Il suffit de montrer que $N(h^2)/N(h) \to 0$ quand $h \to 0$ et $h \neq 0$. En effet $N(h^2) = N(N(h)\epsilon(h)) = N(h)N(\epsilon(h))$ par homogénéité. De plus, par définition, $N(\epsilon(h)) \to 0$ quand $h \to 0$.

Or si \[
h(X) = a_0 + a_1 X + \cdots + a_n X^n
\]
alors
\[
h^2(X) = a_0^2 + 2a_0a_1 X + (2a_0a_2 + a_1^2)X^2 + \cdots + b_k X^k + \cdots + a_n^2 X^{2n}
\]
où on peut montrer par récurrence que 
\[
b_k = \sum_{i+j = k} a_i a_j
\]
Dès lors
\[
N(h^2) =\sup_{k \in \N} \vert b_k \vert = \sup_{k \leq 2n} \vert b_k \vert = \sup_{k \leq 2n}  \left\vert \sum_{i+j = k} a_i a_j\right\vert  \leq \sup_{k \leq 2n}  \sum_{i+j = k} \vert a_i \vert\vert a_j \vert
\]
or pour $i_0$ fixé, $\vert a_{i_0} \vert \leq \sup_i \vert a_i\vert = N(h)$ donc on peut écrire
\[
N(h^2) \leq \sup_{k \leq 2n}  \sum_{i+j = k} N(h)N(h) \leq \sup_{k \leq 2n}  kN(h)N(h) \leq 2n N(h)^2
\]
Ce qui suffit pour conclure.
\end{document}